\documentclass[9pt]{beamer}					% Document class

\usepackage[english]{babel}				% Set language
\usepackage[utf8x]{inputenc}			% Set encoding
\usepackage[ruled,linesnumbered]{algorithm2e}

\mode<presentation>						% Set options
{
  \usetheme{AnnArbor}					% Set theme
  \usecolortheme{orchid} 				% Set colors
  \usefonttheme{default}  				% Set font theme
  \setbeamertemplate{caption}[numbered]	% Set caption to be numbered
}




% Uncomment this to have the outline at the beginning of each section highlighted.
\AtBeginSection[]
{
  \begin{frame}{Outline}
    \tableofcontents[currentsection]
  \end{frame}
}

\AtBeginSubsection[]
  {
     \begin{frame}{Outline}
     \tableofcontents[currentsubsection]
     \end{frame}
  }


\usepackage{emoji}
\usepackage{fontawesome5}
\usepackage{tikz}
\usetikzlibrary{positioning}
\usepackage{subcaption}
\usepackage{graphicx}					% For including figures
\usepackage{booktabs}					% For table rules
\usepackage{hyperref}					% For cross-referencing
\hypersetup{
    colorlinks=true,
}

%\usepackage{enumerate}
\usepackage[shortlabels]{enumitem}

\title{Pseudoentropy}	% Presentation title
\subtitle{PhD Dissertation Talk}
%\author{MaciejSkorski}							% Presentation author
\institute{University of Warsaw}					% Author affiliation
\date{May 23, 2023}									% Dissertation's date	


\begin{document}

% Title page
% This page includes the informations defined earlier including title, author/s, affiliation/s and the date
\begin{frame}
  \titlepage
\end{frame}

% Outline
% This page includes the outline (Table of content) of the presentation. All sections and subsections will appear in the outline by default.

\begin{frame}{This talk}
\begin{itemize}
    \item[\emoji{check-mark}] Overviews the goals, resources, and deliverables of my PhD project.
    \item[\emoji{check-mark}] Demonstrates/sketches interesting techniques used in the dissertation.
    \item[\emoji{check-mark}] Defends my position on the dissertation form, in view of reviews received \emoji{shield}.
    \item[\emoji{cross-mark}] Avoids complex definitions and proofs for brevity's sake (see the papers) \emoji{stopwatch}.
    \item[\emoji{cross-mark}] Does not assess my own academic KPIs (see the documentation) \emoji{flushed-face}.
\end{itemize}
\end{frame}

\begin{frame}{Outline}
  \tableofcontents
\end{frame}

% The following is the most frequently used slide types in beamer
% The slide structure is as follows:
%
%\begin{frame}{<slide-title>}
%	<content>
%\end{frame}

\section{Acknowledgments \emoji{folded-hands}}

\begin{frame}{Credits}
I am particularly grateful:
\begin{itemize}
    \item \emoji{red-heart} for love, to my wife Aneta
    \item \emoji{money-bag} for funding and know-how, to my advisor Stefan Dziembowski
    \item \emoji{light-bulb} for merit support, to my co-advisor Krzysztof Pietrzak
    \item \emoji{clapping-hands} for motivation and recognition, to dozens of people with whom I shared ideas: research collaborators, reviewers, audience of my talks \emoji{smiling-face}
\end{itemize}
\end{frame}

\begin{frame}{Funding}
My PhD research received support from numerous funding sources:
\newline
%\setlist[itemize]{align=parleft}
\begin{itemize}[leftmargin=5em]
    \item[{\includegraphics[height=15pt]{figures/Foundation_for_Polish_Science_logo.jpg}}] \qquad Ideas for Poland
    \item[{\includegraphics[height=15pt]{figures/Foundation_for_Polish_Science_logo.jpg}}]\qquad WELCOME
    \item[{\includegraphics[height=15pt]{figures/logo-erc.jpg}}]\qquad TOCNeT
    \item[{\includegraphics[height=15pt]{figures/logo-ncn.png}}]\qquad PRELUDIUM
    \item[{\emoji{globe-showing-americas}}]\qquad + several travel grants from various research institutions
\end{itemize}
\end{frame}

\section{Introduction \emoji{chequered-flag}}



\begin{frame}{About Pseudoentropy}
\begin{itemize}[leftmargin=3em]
\item[{\emoji{woman-scientist}\emoji{man-scientist}}] Introduced in~\cite{Impagliazzo1989,DBLP:journals/siamcomp/HastadILL99} as a \textbf{computational variant of information-theoretic entropy}.

\item[\emoji{gear}\emoji{hammer}] Recognized as a \textbf{useful tool and convenient language} in research around
cryptography, computational complexity and information theory. Examples:
\begin{itemize}
    \item[\emoji{bookmark-tabs}] Pseudorandom generators from one-way functions \cite{DBLP:journals/siamcomp/HastadILL99}
    \item[\emoji{bookmark-tabs}] Computational Dense
Model Theorem \cite{DBLP:conf/focs/ReingoldTTV08,DBLP:journals/eccc/Zhang11}, improving upon the key ingredient in the results of Green-Tao-Ziegler~\cite{green2008primes,tao2008primes}.
\end{itemize}
\item[\emoji{puzzle-piece}\emoji{question}] Promising but messy: suffers from \textbf{contextual definitions} and \textbf{insufficiently developed foundations}.
% https://emoji-copy.com/en/pile-of-poo-emoji-copy-paste/
\end{itemize}
\end{frame}

\begin{frame}{Goals}
	My PhD project set these goals:
    \begin{itemize}
	\item[\emoji{broom}] \textbf{improve understanding of foundational properties} of pseudoentropy notions
        \item[\emoji{gear}] \textbf{demonstrate further technical applications}
        \item[\emoji{gem-stone}] optionally, identify \textbf{new inspirational application areas}
	\end{itemize}
\end{frame}

\begin{frame}{Contribution}
Works presented under the scope of this PhD project:
\begin{itemize}
    \item [\emoji{heavy-check-mark}] \textbf{obtained characterizations and manipulation rules} for pseudoentropy notions, using \textbf{convex analysis as a toolbox}
    \item [\emoji{heavy-check-mark}] \textbf{simplified some of existing technical proofs}, for instance of Dense Model Theorem and of Computational Simulators
    \item [\emoji{heavy-check-mark}] \textbf{developed machine-learning inspired framework} for proving computational indistinguishability
\end{itemize}
My self-assesment:
\begin{itemize}
\item[\emoji{goal-net}] these works contributed to the goals 
\emoji{broom},\emoji{gear} and \emoji{gem-stone} respectively.
\item[\emoji{person-running}] goals were set broadly, leaving still room for improvement
\end{itemize}
\end{frame}


\section{Detailed Overview \emoji{magnifying-glass-tilted-right}}

\subsection{Preliminaries}

\begin{frame}{Background}
\newcommand{\D}{\mathsf{D}}
\begin{itemize}[\emoji{key}]
\item Pseudoentropy at least $k$ when the distribution behaves \emph{nearly as well} as with information-theoretic (min)entropy $k$ in \emph{cryptographic games}.
\item Program-input games used in definitions
\begin{enumerate}[(a)]
\item Distinguish: discriminate between two distributions based on a sample.\newline  Pseudoentropy example: for a pseudorandom generator $G$ from $d$-bit seeds to $k$-bit outputs, $|\mathbf{E}\D(G(U_d))-\mathbf{E}\D(U_k)|\leqslant \epsilon$ for small $\epsilon$ and comp. bounded $\D$.
\item Predict: guess a sampled outcome. \newline
Pseudoentropy example: a one way function $f$ satisfies $\mathbf{P}\{ f(\D(f(x)))=f(x) \}\leqslant 2^{-k}$ for computationally bounded $\D$ and large $k$.
\item Compress: decoding / encoding games, see~\cite{yao1982theory, DBLP:conf/random/BarakSW03}.
\item ... and even more exotic examples, see~\cite{haitner2009inaccessible}.
\end{enumerate}
\end{itemize}
\end{frame}

\subsection{Geometric Characterizations of Pseudoentropy \emoji{broom}\emoji{gear}}

\begin{frame}{Outline}
\begin{itemize}
\item[\emoji{open-book}] Indistinguishability quantifies how close are two distributions under a given class of computationally bounded tests. 
\item[\emoji{question}] What is the geometrical meaning of indistinguishability?
\item[\emoji{raised-hand}] Computational indistinguishability can be \textbf{characterized by inseparability by a class of feasible hyperplanes}. The separation margin can be determined analytically too! 
\end{itemize}
\end{frame}

\begin{frame}{Contribution}
The characterizations~\cite{DBLP:conf/icits/Skorski15} has found the following applications:
\begin{itemize}
    \item[\emoji{key}] Unifying unpredictability-based and indistinguishability-based pseudoentropy notions~\cite{DBLP:conf/icalp/SkorskiGP15}
    \item[\emoji{key}] Short proof of the Dense Model Theorem~\cite{DBLP:conf/icits/Skorski15a}
    \item[\emoji{key}] New proofs of Hardcore Lemmas~\cite{DBLP:conf/icits/Skorski15a}
    \item[\emoji{key}] Further applications to key derivation~\cite{Skorski17a}
    \item[\emoji{key}] Simplifies other technical arguments~\cite{DBLP:conf/stoc/VadhanZ12}
\end{itemize}
\end{frame}

\begin{frame}{Technique (Sketch)}

\newcommand{\D}{\mathsf{D}}
\newcommand{\Y}{\textcolor{red}{\mathsf{Y}}}

\begin{itemize}[\emoji{key}]
\item In program-input indistinguishability games, it makes sense to \textbf{characterize the optimal input player} $Y$ against a given program player $\D$.
\item View $\D$ as a \emph{separating hyperplane}, \emph{maximize margin} with high-entropy $Y$.
\begin{figure}[htb]
\begin{subfigure}{0.3\textwidth}
\begin{tikzpicture}[
    scale=0.4,
    font=\sffamily,
    bullet/.style=
        {circle, minimum size=1,
        inner sep=0pt, outer sep=0pt
    },nodes=bullet
]
 \node[circle,draw,fill=gray, minimum size=1.25cm] (Y) at (0,0) {$Y$: high entropy} ;
 \node[label={south west:$Y_{\D}^{\max}$}] (Ymax) at (Y.225) {$\bullet$};
 \node[label=$\mathbf{P}_X$] (X) at (3,3) {$\bullet$};
 \draw(0,5)  to node[sloped,midway,below right=0.25] {$\D$}(5,0);
 \draw[->, dashed] (X) -- (Ymax);
\end{tikzpicture}
\end{subfigure}
\hfill
\begin{subfigure}{0.6\textwidth}
\resizebox{0.95\linewidth}{!}{
    \begin{tabular}[b]{ccc}\hline
      Symbol/Operator & Crypto & Geometry\\ \hline
      $X$ & candidate distribution \\
      $Y$ & input player & feasible point \\
      $\D$ & distinguisher/program player & separating hyperplane \\
      $\mathbf{E}\D(Y)$ & expectation & $\D\cdot \mathbf{P}_Y$ (dot-product)\\
      $\epsilon=\mathbf{E}\D(Y)-\mathbf{E}\D(X)$  & advantage & separation margin \\
 \hline
    \end{tabular}
}
\end{subfigure}
\caption{Geometrical meaning of cryptographic indistinguishability.}
\end{figure}
\item \textbf{Closed-form solutions found} in interesting cases by \textbf{convex optimization}. For pseudoentropy of at least $k$ bits against attackers $\mathcal{D}$ with advantage $\epsilon$: $$\forall \D\in\mathcal{D}:\quad \mathbf{E}\D(X) \leqslant 2^{-k}|\D|+\epsilon$$ instead of the standard depth-2 formula $\forall \D \exists Y:\mathbf{H}_{\infty}(Y)\geqslant k \& \ \mathbf{E}\D(X) \leqslant \mathbf{E}\D(Y)+\epsilon$.
\item[\emoji{warning}] Characterization depend on feasible distinguishers and the baseline entropy.
\end{itemize}

\end{frame}

\subsection{Unpredictability Pseudoentropy \emoji{broom}}

\begin{frame}{Outline}
\begin{itemize}
\item[\emoji{open-book}] Applications of pseudoentropy use different notions, most commonly unpredictability-based and indistinguishability-based.
\item[\emoji{question}] Are unpredictability and indistinguishability entropies different? 

Note: usually, distinguishing is easier than predicting\footnote{Think of discriminating between dogs and cats versus predicting the breed.}.
\item[\emoji{raised-hand}] Surprisingly, \textbf{equivalent in high-entropy regimes}!
\end{itemize}
\end{frame}

\begin{frame}{Contribution}
The following result was obtained~\cite{DBLP:conf/icalp/SkorskiGP15}:
\begin{itemize}
    \item[\emoji{key}] \textbf{equivalence of unpredictability and indistinguishability} pseudoentropy definitions in \textbf{high-entropy regimes}, namely $n-O(\log n)$ for $n$-bit strings,
    \item[\emoji{key}] \textbf{geometric characterizations as a workhorse} of the proof.
\end{itemize}
\end{frame}

\begin{frame}{Technique (Sketch)}
The proof strategy is to \emph{constructively convert a distinguisher into a predictor}:
\newcommand{\D}{\textcolor{blue}{\mathsf{D}}}
\begin{enumerate}[(a)]
    \item Indistinguishability fails: $\mathbf{E}\D(X)\geqslant \mathbf{E}\D(Y)+\epsilon$ for all $Y$ of min-entropy $k$.
    \item $\mathbf{E}\D(X)\geqslant |\D|/2^{k}+\epsilon$ for boolean $\D$, by geometrical characterizations (!)
    \item Sample $\textcolor{green}{A}$ from the image of $\D$, then $\mathbf{P}\{\textcolor{green}{A} = X\} > 2^{-k}+\frac{\epsilon}{\#\D}$.
    \item Approximate image sampling by \emph{rejection sampling} $\ell$ times, then
    $$\mathbf{P}\{\textcolor{green}{A} = X\} > \left( 2^{-k}+\frac{\epsilon}{\#\D} \right)\cdot \left(1-\frac{\# \D}{2^n}\right)^{\ell}.$$
    \item $\mathbf{P}\{\textcolor{green}{A} = X\} > 2^{-k}$  when $\ell \approx 2^{\textcolor{red}{n-k}}/\epsilon$ independently of $\# \D$ !
    \item[\emoji{warning}] More sophisticated rejection-sampling handles $X$ with auxiliary input $Z$.
\end{enumerate}
\end{frame}


\note{Everything you want}


\subsection{Best Generic Attacks on Pseudoentropy \emoji{gear}}

\begin{frame}{Outline}
\begin{itemize}
\item[\emoji{open-book}] Applications of pseudoentropy assume strength parameters that propagate through reduction proofs. Not clear what regimes are non-trivial to start with.
\item[\emoji{question}] Can we characterize when quality parameters are non-trivial?
\item[\emoji{raised-hand}] Yes, by time-advantage tradeoffs - similarly to pseudorandomness!
\end{itemize}
\end{frame}

\begin{frame}{Contribution}
The following result was obtained~\cite{Skorski17a}:
\begin{itemize}
    \item[\emoji{key}] generic attacks with time $t$ succeed against psuedoentropy amount $k$ with advantage $\epsilon = O\left(\sqrt{t/2^{k}}\right)$.
    \item[\emoji{key}] generalization of the famous time-advantage tradeoffs against pseudorandomness \cite{DBLP:conf/crypto/DeTT10}.
\end{itemize}
\end{frame}

\begin{frame}{Technique (Sketch)}
\newcommand{\D}{\mathsf{D}}
The proof leverages \textbf{random walk techniques}, see also~\cite{berger_fourth_1997}.
\begin{enumerate}[(a)]
\item Let $\D:\{0,1\}^n\to \{-1,1\}$ be fully random (Rademacher), then $|\mathbf{E} \D(X)-\mathbf{E} \D(Y)| \approx \|\mathbf{P}_X-\mathbf{P}_Y\|_2$ w.h.p. (random walks theory~\cite{haagerup1981best}).
\item By slicing the domain $\{0,1\}^n$ into $t$ random parts and flipping the signs accordingly, we can have
$|\mathbf{E} \D'(X)-\mathbf{E} \D'(Y)| \approx t\cdot \frac{\|\mathbf{P}_X-\mathbf{P}_Y\|_2}{\sqrt{t}}$ w.h.p., with $O(t)$ extra memory.
\item Under mild assumptions on $Y$ (far from having $k$ bits of min-entropy) the attack achieves advantage $\epsilon\approx \sqrt{t/2^k}$. Compare with $\epsilon\approx \sqrt{t/2^n}$ for pseudorandomness!
\item "Random" can be weakened to \textbf{$O(1)$-wise independent}, and the construction complexity is indeed $O(t)$, alternatively complexity $t$ yields $\epsilon=O(\sqrt{t/2^k})$!
\end{enumerate}   
\end{frame}

\subsection{Lower Bounds for Pseudoentropy Chain Rules and Transformations \emoji{gear}}

\begin{frame}{Outline}
\begin{itemize}
\item[\emoji{open-book}] Applications of pseudoentropy \textbf{heavily rely on manipulation rules}, particularly chain rules and transformations~\cite{DBLP:conf/random/BarakSW03,DBLP:conf/tcc/FullerOR12}. However, their use \textbf{weakens security guarantees}, due to tradeoffs in quality parameters \textbf{caused by reduction proofs}.
\item[\emoji{question}] Can we improve known manipulation rules?
\item[\emoji{raised-hand}] No, not by black-box reductions!
\end{itemize}
\end{frame}

\begin{frame}{Contribution}
The following results were obtained in~\cite{pietrzak2016pseudoentropy}:
\begin{itemize}
    \item[\emoji{key}] \textbf{Impossibility of better proofs} by black-box reductions!
    \item[\emoji{key}] A \textbf{probabilistic construction of an oracle}, of independent interest, inspired by the earlier work on limitations of dense model theorems~\cite{DBLP:journals/eccc/Zhang11}.
    \item[\emoji{fire}] Inspired techniques used in research on black-box limitations of auxiliary input simulators~\cite{Chen_2018}.
\end{itemize}
\end{frame}

\begin{frame}{Techniques (Sketch)}
\newcommand{\D}{\mathsf{\D}}
\begin{enumerate}[(a)]
    \item Proofs, in case of indistinguishability-based pseudoentropy, rely on \emph{building distinguishers from distinguishers in other setups}. In particular, we have 
    $D' = \mathbb{I}\{\sum_i w_i D_i > t_0\}$  in the proof of the Dense Model Theorem~\cite{DBLP:journals/eccc/Zhang11} or transformations~\cite{DBLP:conf/random/BarakSW03,skorski2015new}.
    \item Loosely speaking, black-box reductions \emph{aggregate distinguishers by high-level operations}. To prove the need of many operations (queries) we manipulate distinguishers at low-level by \emph{choosing values probabilistically and sophistically}!
    \item Examples (from the proof, see also~\cite{Chen_2018}): 
    \begin{enumerate}[(1)]
    \item $D_i\sim \mathsf{Bern}(1/2+\epsilon)$ on a small set and $D_i\sim \mathsf{Bern}(1/2)$ elsewhere.
    \item $D_i\sim \mathsf{Bern}(1/2+\epsilon)$ and $D_i\sim \mathsf{Bern}(1/2-\epsilon)$ on complementary random subsets.
    \end{enumerate}
    \item[\emoji{warning}] High-level aggregations are essential. For example, without them Dense Model Theorems can fail~\cite{impagliazzo2020comparing}.
\end{enumerate}
\end{frame}

\subsection{Simulating Auxiliary Information \emoji{gem-stone}}

\begin{frame}{Outline}
\begin{itemize}
\item[\emoji{open-book}] In security proofs, it helps to model leakages as explicit functions of secrets~\cite{DBLP:conf/tcc/JetchevP14}.
\item[\emoji{question}] What leakages can be modelled as functions of secrets?
\item[\emoji{raised-hand}] Short leakages can be efficiently simulated!
\end{itemize}
\end{frame}

\begin{frame}{Contribution}
The following important results were obtained~\cite{skorski2016simulating}:
\begin{itemize}
\item[\emoji{key}] Construction of a simulator for $m$ bits of leakage which makes only $2^{O(m)}\epsilon^{-2}$ calls to achieve $\epsilon$-indistinguishability. Significantly improved upon prior works~\cite{DBLP:conf/crypto/VadhanZ13,DBLP:conf/tcc/JetchevP14}
\item[\emoji{trophy}] The reasoning, inspired by ML techniques, \textbf{builds on the gradient descent algorithm} and was recognized with the \emph{best student paper award at TCC'16}.
\item[\emoji{fire}] Inspired follow-up works that solved the simulator problem~\cite{Chen_2018}, and generalized the learning framework~\cite{skorski2016subgradient,skorskiapproximating}, and studied quantum pseudoentropy (c.f. Chen's dissertation~\cite{chen2019computational}). 
\end{itemize}
\end{frame}

\begin{frame}[fragile]{Technique (Sketch)}
\begin{enumerate}[(a)]
\item The algorithm below demonstrates the procedure  
\newcommand{\D}{\mathsf{D}}
\scalebox{0.75}{
\begin{algorithm}[H]
\SetAlgoLined
\DontPrintSemicolon
\caption{Auxiliary Input Simulator}\label{alg:two}
\KwData{
\begin{enumerate}[-]
\item Oracle access to distinguishers/test functions $\mathcal{D}$
\end{enumerate}
}
\KwResult{Simulator $h$ of $Z\in\{0,1\}^m$ given $X\in\{0,1\}^n$}
$\mathbf{P}\{h(x)=z\}\gets 2^{-m}$ \tcp*{initialize the solution as uniform} 
\While{$\max_{D\in\mathcal{D}} \mathbf{E}\D(X,Z)-\mathbf{E}\D(X,h(X)) > \epsilon$ \tcp*{as long as can distinguish...} }{
  $\mathbf{P}\{h'(x)=z\} \gets \mathbf{P}\{h(x)=z\} - \gamma \D(x,z)$ \tcp*{improve candidate}
  $\mathbf{P}\{h'(x)=z\} \gets \mathbf{P}\{h'(x)=z\} + \mathsf{Correct}(x,z)$ \tcp*{guarantee constraints}
  $h\gets h'$
 }
\Return $h$
\end{algorithm}
}
\item Outputs an efficient simulator p.d.f., with appropriate  $\gamma$ and $\mathsf{Correct}$ operation. 
\item Finishes after $2^{O(m)}\epsilon^{-2}$ steps, proved by "energy" arguments.
\item Resembles boosting: we learn how to (strongly) simulate from (weak) distinguishers
\item Resembles convex optimization: with $\D$ as subgradient, $\gamma$ as a stepsize, $\mathsf{Correct}$ as a projection operation!
\end{enumerate}
\end{frame}


%\bibliography{citations}

\section{References \emoji{books}}

% Adding the option 'allowframebreaks' allows the contents of the slide to be expanded in more than one slide.
\begin{frame}[allowframebreaks]{References}
	\tiny\bibliography{citations}
	\bibliographystyle{alpha}
\end{frame}

\section{Discussion \emoji{speech-balloon}}

\begin{frame}{Addressing Reviewers Feedback}
\begin{itemize}
    \item[R:] Editorial changes and reference requests.
    \item[M:] Addressed, thanks for the feedback!
    \item[R:] A book-style dissertation would be better than a mixture of conference works.
    \item[M:] I discussed this form with senior researchers, but found \emph{ineffective}:
    \begin{itemize}
        \item[\emoji{carrot}] Gain citations! \emoji{thinking}\emph{Time-consuming, better to keep writing papers.}
        \item[\emoji{carrot}] Get your PhD distinguished. \emoji{thinking}\emph{Prestigious conferences not enough?}
        \item[\emoji{sweat-droplets}] Take your time to present it better! \emoji{thinking}\emph{Why to work harder? We count conference works when granting junior/senior  professorships!} 
    \end{itemize}
    \item[R:] Parts of lengthy works might not have been fully reviewed at conferences.
    \item[M:] Same can happen for junior professorships, but we had extra reviewers \emoji{relieved-face}.
\end{itemize}
\end{frame}

\end{document}